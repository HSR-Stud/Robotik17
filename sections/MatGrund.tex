\section{Mathematische Grundlagen der Robotik}

	\subsection{TI Taschenrechner Befehle}
\begin{tabular}{p{5cm}p{10cm}}

\texttt{$[ x_{11} , x_{12} ; x_{21},x_{22}]$} &  $ \begin{bmatrix}
            	x_{11} & x_{12}\\
            	x_{21} & x_{22}\\
            \end{bmatrix} $ \\
\texttt{ $[ \ldots]^{-1} $} & Inverse Matrix\\
\texttt{ $[ \ldots] $ (2ND CATALOG) \small{T} } & Transponierte Matrix
$[\ldots]^{T}$\\

\end{tabular} \\
	

	\subsection{Matritzenrechnen}
		\subsubsection{Vektoren im Raum}
			$\vec{p}_{AB}=
			\begin{matrix}
            	x_B-x_A\\
            	y_B-y_A\\
            	z_B-z_A\\
            \end{matrix}$\\
			
			Transponierung: $a^T\cdot b= \left(a_1 \vspace{0.2cm} a_2\vspace{0.2cm}  \ldots a_n \right) 
			\cdot $
	
		\subsubsection{Matrizen Multiplikation \small{(A muss gleich viele Spalten
		haben wie B Zeilen hat)}}
		\includegraphics[width=7cm]{./bilder/matrizenmultiplikation.png}
    	
	
	
	
\subsection{Übersicht}
	\begin{tabular}{l l}
    	Transponierte Matrix: & $A^T=[a_{ik}^T]=[a_{ki}]$ vertauschen der Zeilen
    	mit Spalten\\
    	Einheitsmatrix:& $I_n= 
			    	\begin{bmatrix} 
			        	1&0 & 0\\
			        	0&1&0\\
			        	0&0&1                               
			        \end{bmatrix}$		    
    \end{tabular}

\subsection{Determinante}

	\textbf{2x2 Matrix}    
	$$ \det \begin{bmatrix} a_{11} & a_{12} \\ a_{21} & a_{22} \end{bmatrix} =
	a_{11} a_{22} - a_{12} a_{21}.  $$
	
	\textbf{3x3 Matrix}
	$$ \det \begin{bmatrix} a_{11} & a_{12} & a_{13} \\ a_{21} & a_{22}& a_{23} \\
	a_{31} & a_{32} & a_{33} \end{bmatrix} \\ = a_{11} a_{22} a_{33} + a_{12}
	a_{23} a_{31} + a_{13} a_{21} a_{32} - a_{13} a_{22} a_{31} - a_{12} a_{21}
	a_{33} - a_{11} a_{23} a_{32}.  $$
	
% 	\textbf{Dreiecksmatrix} - Alle Elemente entweder ober- oder unterhalt der Hauptdiagonale $= 0$
% 	$$\det A =a_{11}\cdot a_{22}\dotsb a_{nn} \quad  \quad \text{Die Det. ist das Produkt
% 	der Hauptdiagonal-Einträge. Gilt somit auch für Diagonalmatritzen.} $$
% 	
	\textbf{Null $(|A| = 0)$} - Wenn $A$ eine (n,n)-Matrix ist, so wird $|A| = 0$ unter einer der
	folgenden Bedingungen:
	\begin{itemize}
    	\item Zwei Zeilen/Spalten sind linear abhängig (gleich oder ein Vielfaches der anderen).
    	\item Alle Elemente einer Zeile/Spalte sind Null. \\
  	\end{itemize} 
	
% 	\textbf{Allgemein:}
% 	$$A\epsilon M_n: \det A =    
% 	\begin{vmatrix}
%     	a_{11} & a_{12}& \ldots & a_{1n}\\
%     	a_{21}& &\ldots & \\
%     	\ldots \\
%     	a_{n1} & & \ldots & a_{nn}    			
%     \end{vmatrix}=
% 	(-1)^{1+1}a_{11}D_{11} + (-1)^{1+2}a_{12}D_{12}+ \ldots +
% 	(-1)^{1+n}a_{1n}D_{1n}$$
% 	
% 	\subsubsection{Unterdeterminante}
% 	$$D_{11}=
% 	\begin{vmatrix}
%     	a_{22} & \ldots & a_{2n}\\
%     	\ldots\\
%     	a_{n2}& \ldots & a_{nn}
%     \end{vmatrix} 	\\
% 	D_{12}=
% 	\begin{vmatrix}
%     	a_{21} & a_{23}& \ldots & a_{2n}\\
%     	\ldots\\
%     	a_{n1}& a_{n3}&\ldots & a_{nn}
%     \end{vmatrix}$$\\
% 	$D_{ij}$ die (n-1)$ \times $(n-1)-Untermatrix von D ist, die durch Streichen der
% 	i-ten Zeile und j-ten Spalte entsteht.\\
% 	Diese Methode ist zu empfehlen, wenn die Matrix in einer Zeile oder Spalte
% 	bis auf eine Stelle nur Nullen aufweisst.
% 	Dies lässt sich meist mit dem Gausverfahren bewerkstelligen.
% 	
% \subsection{Gaussverfahren}
% 	Durch Addition und Subtraktion einzelner Zeilen (auch von Vielfachen einer
% 	Zeile) werden einzelne Stellen auf Null gebracht. zB:\\
% 	$\begin{bmatrix}
%     	a_{11} & a_{12}& \ldots & a_{1n}\\
%     	a_{21}& &\ldots & \\
%     	\ldots \\
%     	a_{n1} & & \ldots & a_{nn}    			
%     \end{bmatrix}=
% 	\begin{bmatrix}
%     	a_{11} & a_{12}& \ldots & a_{1n}\\
%     	k a_{21}-n a_{11}& ka_{22}-n a_{12}&\ldots & k a_{2n} - n a_{1n}\\
%     	\ldots \\
%     	a_{n1} & & \ldots & a_{nn}    			
%     \end{bmatrix}$ \\
% 	Die n * erste Zeile wurde von der k * zweiten Zeile abgezogen ($a_{2.}= 
% 	k a_{2.}- n a_{1.}$) 
	
\subsection{Inverse Matrix \small{(Existiert nur wenn Matrix regulär: $\det A \neq 0$)}}
\begin{minipage}{7cm}
	\textbf{2x2 Matrix:}    
	$$ A^{-1} = \begin{bmatrix} a & b \\ c & d \\ \end{bmatrix}^{-1} = \frac{1}{ad
	- bc} \begin{bmatrix} d & -b \\ -c & a \\ \end{bmatrix} $$
\end{minipage}
\begin{minipage}{11cm}
	\textbf{3x3 Matrix:}
  $$  A^{-1} = \begin{bmatrix} a & b & c\\ d & e & f \\ g & h & i \\ \end{bmatrix}^{-1} =
  \frac{1}{\det(A)} \begin{bmatrix} ei - fh & ch - bi & bf - ce \\ fg - di & ai
  - cg & cd - af \\ dh - eg & bg - ah & ae - bd \end{bmatrix} $$
\end{minipage}\\

% \textbf{Diagonalmatrix} (Alle Elemente ausserhalb der Hauptdiagonale $= 0$, Elemente auf
% Hauptdiagonale sind Eigenwerte $\lambda_i$): \\ 
% Alle Elemete elementweise invertieren - Kehrwert. $\quad \Rightarrow \quad $\textit{Gilt nur wenn
% alle Elemente auf der Hauptdiagonale $\neq 0$ sind.}\\
% 
% \textbf{Allgemein:}\\
% 	$A^{-1}= \begin{bmatrix}
%     	a_{11} & a_{12}& \ldots & a_{1n}\\
%     	a_{21}& &\ldots & \\
%     	\ldots \\
%     	a_{n1} & & \ldots & a_{nn}    			
%     \end{bmatrix}^{-1}$
% 	\begin{enumerate}
% 		\item $A^T$ bestimmen (Zeilen und Spalten vertauschen) $A^{T}= \begin{bmatrix}
%     	a_{11} & a_{21}& \ldots & a_{n1}\\
%     	a_{12}& &\ldots & \\
%     	\ldots \\
%     	a_{1n} & & \ldots & a_{nn}    			
%     \end{bmatrix}$	
% 		\item Bei $A^T$ jedes Element $a_{ij}$ durch Unterdet. $D_{ij}$ mit
% 		richtigem Vorzeichen ersetzen $A^*=	\begin{bmatrix}
% 			(-1)^{1+1}D_{11} &  \ldots	& (-1)^{1+n} D_{1n}\\
% 			\ldots\\
% 			(-1)^{n+1} D_{n1}& \ldots  & (-1)^{n+n} D_{nn}
% 		\end{bmatrix}$
% 		\item $A^{-1} = \frac{A^*}{\det A}$ 
%     \end{enumerate}
%  
%  \subsection{Diagonalisierung}
%  	\begin{enumerate}
%        \item Eigenwerte $\lambda$ auschrechnen: $\det (A - I_n \lambda)=0$
%        \item Eigenvektoren $\vec{v}$ bilden: $(A- \lambda I_n)\vec{v}=0$
%        \item Transformationsmatrix: $T= [\vec{v_1} \ldots \vec{v_n}]$
%        \item $T^{-1}$ berechnen (Achtung ist A symmetrisch, dh. $A^T=A$ und
%        oder alle EV senktrecht zueinander, dann $T^{-1}=T^T$)
%        \item $D=\begin{bmatrix}
%                 	\lambda_1 &0 &0\\
%                 	0& \lambda_2 &0\\
%                 	0& 0& \lambda_3
%                 \end{bmatrix}$
% 		\item $A^n = T D^n T^{-1}$
% 
%      \end{enumerate}

\subsection{Basis Rotationsmatrizen}
	\begin {minipage}{7cm}
		Rotation um die x-Achse\\ \\
   		$R_z(\alpha)=\begin{bmatrix}
                		1 &0 &0\\
                		0 &cos\theta &-sin\theta\\
                		0 &sin\theta &cos\theta
                	 \end{bmatrix}$
    \end{minipage}
	\begin{minipage}{7cm}
    	Rotation um die y-Achse\\ \\
   		$R_z(\alpha)=\begin{bmatrix}
                		cos\beta &0 &sin\beta\\
                		0 &1 &0\\
                		-sin\beta &0 &cos\beta
                	 \end{bmatrix}$
    \end{minipage}
	\begin{minipage}{7cm}
    	Rotation um die z-Achse\\ \\
   		$R_z(\alpha)=\begin{bmatrix}
                		cos\alpha &-sin\alpha &0\\
                		sin\alpha &cos\alpha &0\\
                		0 &0 &1
                	 \end{bmatrix}$
    \end{minipage}

\subsection{Aufeinanderfolgende Rotationen}
	3 Koordinatensysteme \{A\}, \{B\}, \{C\} mit gleichem Ursprung\\ \\
	\begin{minipage}{10cm}
		Körperfestkoordinatensystem\\ \\
		${}^B\mathrm{p}={}^B_C\mathrm{R}\cdot{}^C\mathrm{p}$\\
		${}^A\mathrm{p}={}^A_B\mathrm{R}\cdot{}^B\mathrm{p}$\\
		${}^A\mathrm{p}={}^A_C\mathrm{R}\cdot{}^C\mathrm{p}$\\ \\
		${}^A_C\mathrm{R}=\overrightarrow{{}^A_B\mathrm{R} \cdot {}^B_C\mathrm{R}}$\\
	\end{minipage}
	\begin{minipage}{10cm}
		Raumfestkoordinatensystem\\ \\
		${}^B\mathrm{p}={}^B_A\mathrm{R}{}^B_C\mathrm{R}{}^A_C\mathrm{R}\cdot{}^C\mathrm{p}$\\
		${}^A\mathrm{p}={}^A_B\mathrm{R}\cdot{}^B\mathrm{p}$\\
		${}^A\mathrm{p}={}^A_C\mathrm{R}\cdot{}^C\mathrm{p}$\\ \\
		${}^A_C\mathrm{R}=\overleftarrow{{}^B_C\mathrm{R} \cdot {}^A_B\mathrm{R}}$\\
	\end{minipage}

\subsection{Konvention Arctan2}
	\begin{minipage}{10cm}
    	$\Theta=\arctan2(y,x)\Longrightarrow\left\{
    	\begin{array}{l}
            \text{1:\space\space}\Theta=\arctan(\frac{y}{x})\\
			\text{2:\space\space}\Theta=\pi+\arctan(\frac{y}{x})\\
			\text{3:\space\space}\Theta=-\pi+\arctan(\frac{y}{x})\\
			\text{4:\space\space}\Theta=-\arctan(\frac{y}{x})\\
		\end{array}\right.$
    \end{minipage}
	\begin{minipage}{10cm}
    \includegraphics[width=4cm]{./bilder/einheitskreis.png}
    \end{minipage}






	\begin{minipage}{4cm}
    	\vspace{8mm}
    	$\theta=arctan2(\overbrace{y}^{sin},\overbrace{x}^{cos}) \Leftrightarrow$
    \end{minipage}
	\begin{minipage}[t]{0.8cm}
    	\textcircled{1}\\
    	\textcircled{2}\\
    	\textcircled{3}\\
    	\textcircled{4}
    \end{minipage}
	\begin{minipage}[t]{3.5cm}
    	$\theta= arctan(\frac{y}{x})$\\
    	$\theta= \pi + arctan(\frac{y}{x})$\\
    	$\theta= -\pi + arctan(\frac{y}{x})$\\
    	$\theta= -arctan(\frac{y}{x})$\\
    \end{minipage}
	\begin{minipage}[t]{2.5cm}
    	\hspace{2mm} $0 \leq \theta \leq \frac{\pi}{2}$\\
    	\hspace*{1.5mm} $\frac{\pi}{2} \leq \theta \leq \pi$\\
    	$-\pi \leq \theta \leq -\frac{\pi}{2}$\\
    	$-\frac{\pi}{2} \leq \theta \leq 0$
    \end{minipage}
	\begin{minipage}[t]{2.5cm}
    	für $+y$, $+x$\\
    	für $+y$, $-x$\\
    	für $-y$, $-x$\\
    	für $-y$, $+x$
    \end{minipage}
	\begin{minipage}[t]{4cm}
     	\begin{tabular}[t]{|l}
    	$\theta = \frac{\pi}{2}$ für $x=0$, $y>0$\\
    	$\theta = - \frac{\pi}{2}$ für $x=0$, $y<0$\\
    	$\theta = 0$ für $x=0$, $y=0$\\
    	\end{tabular}
    \end{minipage}

$Arctan2$ mit TR: \texttt{$R \blacktriangleright P \theta (X,Y)$}
\hspace{2cm} {\color{red} ACHTUNG: X und Y sind vertauscht!!!}


\subsection{Rücktransformation auf Winkel}
	\begin{minipage}{9.5cm}
    	\begin{tabular}{|p{2.5cm}|p{6cm}|}
        \hline
        	\multicolumn{2}{|l|}{\textbf{Raumfestes Koordinatensystem}}\\
        	\multicolumn{2}{|l|}{(X-Y-Z Roll-Gier-Nick Winkel)}\\
        \hline
    		Rotationsmatrix:
    		& ${^A_B}R(\alpha,\beta,\gamma) = 
    			\begin{bmatrix} 
			    	r_{11} & r_{12} & r_{13} \\
			        r_{21} & r_{22} & r_{23} \\
			        r_{31} & r_{32} & r_{33}                              
			    \end{bmatrix}$ \\
		\hline
			Ansatz:
			& $cos\beta = \sqrt{r^2_{11} + r^2_{21}}$ \\
			& $sin\beta = -r_{31}$\\
		\hline
			Winkel:
			& $\beta=arctan2(-r_{31},\sqrt{r^2_{11}+r^2_{21}})$\\
			& $\alpha=arctan2(\frac{r_{21}}{cos\beta},\frac{r_{11}}{cos\beta})$\\
			& $\gamma=arctan2(\frac{r_{32}}{cos\beta},\frac{r_{33}}{cos\beta})$\\
		\hline
			wenn $\beta=90^{\circ}$:
			& $\alpha=0^{\circ},\gamma=arctan2(r_{12},r_{22})$\\
			wenn $\beta=-90^{\circ}$:
			& $\alpha=0^{\circ},\gamma=-arctan2(r_{12},r_{22})$\\
		\hline
        \end{tabular}
    	
    \end{minipage}
	\begin{minipage}{9.5cm}
    	\begin{tabular}{|p{2.5cm}|p{6cm}|}
        \hline
        	\multicolumn{2}{|l|}{\textbf{Körperfestes Koordinatensystem}}\\
        	\multicolumn{2}{|l|}{(Z-Y-Z Euler Winkel)}\\
        \hline
        	Rotationsmatrix:
    		& ${^A_B}R(\alpha,\beta,\gamma) = 
    			\begin{bmatrix} 
			    	r_{11} & r_{12} & r_{13} \\
			        r_{21} & r_{22} & r_{23} \\
			        r_{31} & r_{32} & r_{33}                              
			    \end{bmatrix}$ \\
		\hline
			Ansatz:
			& $sin\beta = \sqrt{r^2_{31} + r^2_{32}}$ \\
			& $cos\beta = r_{33}$\\
		\hline
			Winkel:
			& $\beta=arctan2(\sqrt{r^2_{31}+r^2_{32}},r_{33})$\\
			& $\alpha=arctan2(\frac{r_{23}}{sin\beta},\frac{r_{13}}{sin\beta})$\\
			& $\gamma=arctan2(\frac{r_{32}}{sin\beta},\frac{r_{31}}{sin\beta})$\\
		\hline
			wenn $\beta=0^{\circ}$:
			& $\alpha=0^{\circ},\gamma=arctan2(-r_{12},r_{11})$\\
			wenn $\beta=180^{\circ}$:
			& $\alpha=0^{\circ},\gamma=-arctan2(r_{12},-r_{11})$\\
		\hline
        \end{tabular}
    	

    	
    \end{minipage}
	
	\newpage
	\subsection{Transformations Matrix}
	\subsubsection{Aufbau \small{ (RotationsMatrix und Verschiebung in einer
	Matrix)}}
		\begin{minipage}{10cm}
			Transformationsmatrix: T = $ 
    			\begin{bmatrix} 
			    	r_{11} & r_{12} & r_{13} & o_{ab} \\
			        r_{21} & r_{22} & r_{23} & o_{ab} \\
			        r_{21} & r_{22} & r_{23} & o_{ab} \\
			        0 & 0 & 0 & x                              
			    \end{bmatrix}$
		\end{minipage}
		\begin{minipage}{10cm}
    			x = 1 bei Ortsvektor \\
				x = 0 bei Feiem Vektor
		\end{minipage}
	\subsubsection{Multiplikation}
		Transformations Matrix über mehrere Koordinatensysteme:\\
		
		${}^0_n\mathrm{T}={}^0_1\mathrm{T}\cdot{}^1_2\mathrm{T}\cdot{}$\ldots$\cdot{}^{n-1}_n\mathrm{T}$;
		\space\space\space\space ${}^A_B\mathrm{T} \Rightarrow$
		Transformationsmatrix von Koordinatensystem A nach B
    	
	\subsubsection{Punkte in verschiedenen Koordinatensystemen}
		${}^B\mathrm{P}={}^B_A\mathrm{T}\cdot{}^A\mathrm{P}$ \\ \\
		${}^A\mathrm{P}={}^B_A\mathrm{T}^{-1}\cdot{}^B\mathrm{P}$
		
		
		

	\subsection{Jacobi Matrix \small{ (erklärt durch Beispiel)}}
		
		
        Die Jacobi-Matrix eines Roboterarms beschreibt die Abbildung von
        Gelenkgeschwindigkeiten auf die Lineargeschwindigkeit des TCP
        und die zeitlichen Änderungen der Orientierung des End-Effektors
        bezogen auf ein Referenzkoordinatensystemk z.B. auf das
        Basiskoordinatensystem O auf. \\
        In der Positionsbeschreibung werden alle Parameter die einen Einfluss 
       auf den Greifer haben aufgestellt und dann für die Jakobi-Matrix nach diesen Partiell abgeleitet. \\
    
	     \begin{minipage}{5cm}
	        Für die Jacobi Matrizen \\
	        empfehlen sich \\
	        Kurzschreibweisen
	        \end{minipage}
			\begin{minipage} {8cm}
            
	        $ c_{12}=cos(\theta_{1} + \theta_{2}) $ \\
	        $ s_{12}=sin(\theta_{1} + \theta_{2}) $
	        
	        \end{minipage}\\ \\
	     \begin{minipage}{12cm}
	     \subsubsection{Vorwärtskinematik}
	     	Geg: Gelenkkoordinaten und Geschwindigkeiten: $q ; \dot{x}$ \\
			Ges: Geschwindigkeit des Endeffektors: $\dot{X} = [\dot{x} \dot{y} \dot{z}
			\dot{\alpha} \dot{\beta} \dot{\gamma}]^{T}$ \\ 
			Lsg: Jacobi-Matrix $ \Longrightarrow \dot{X}=J(q)*\dot{q}$ \\ \\
			Positionsbeschreibung des Endeffektors: \\
	     $
	    			\begin{bmatrix} 
				    	x_{e} \\
				        y_{e} \\                             
				    \end{bmatrix}
					=
					\begin{bmatrix} 
				    	d_{2}cos\theta_{1}+l_{3}cos(\theta_{1}+\theta_{3}) \\
				        d_{2}sin\theta_{1}+l_{3}sin(\theta_{1}+\theta_{3}) \\                              
				    \end{bmatrix} $ und $ \Phi_{e}=\theta_{1}+\theta_{3} \\ \\$
			Jacobi-Matrix: \\ $
				J = 
					\begin{bmatrix} 
				    	\frac{\partial{x_{e}}}{\partial{\theta_{1}}} & 
				    	\frac{\partial{x_{e}}}{\partial{d_{2}}} & 
				    	\frac{\partial{x_{e}}}{\partial{\theta_{3}}} \\
				    	\frac{\partial{y_{e}}}{\partial{\theta_{1}}} & 
				    	\frac{\partial{y_{e}}}{\partial{d_{2}}} & 
				    	\frac{\partial{y_{e}}}{\partial{\theta_{3}}} \\
				    	\frac{\partial{\Phi_{e}}}{\partial{\theta_{1}}} & 
				    	\frac{\partial{\Phi_{e}}}{\partial{d_{2}}} & 
				    	\frac{\partial{\Phi_{e}}}{\partial{\theta_{3}}} \\ 
				    \end{bmatrix} \\
			$ \space  $	=
					\begin{bmatrix} 
					    	-d_{2}sin(\theta_{1})-l_{3}sin(\theta_{1}+\theta_{3}) & 
					    	cos(\theta_{1}) & 
					    	-l_{3}sin(\theta_{1}+\theta_{3}) \\
					        d_{2}cos(\theta_{1})+l_{3}cos(\theta_{1}+\theta_{3}) & 
					        sin(\theta_{1}) & 
					        l_{3}cos(\theta_{1}+\theta_{3}) \\
					        1 & 
					        0 & 
					        1 \\                            
					    \end{bmatrix}$		
			\end{minipage}
			\begin{minipage}{8cm}
			\includegraphics[width=5cm]{./bilder/jacobi-bsp.png}
			\end{minipage}
			
			\subsubsection{Rückwertskinematik}
			Geg: Geschwindigkeiten des Endeffektors: $\dot{x}$ \\
			Ges: Gelenkgeschwindigkeiten: $\dot{q}$ \\
			Lsg: Jacobi-Matrix $ \Longrightarrow \dot{q}=J^{-1}\dot{x}$ \\ \\
			Singularität: \\
			Die Jacobi-Matrix kann in singulären Stellungen
			nicht invertiert werden (d.h. die Determinante von J ist 0) und der Roboter
			kann in bestimmten Richtungen keine Bewegungen
			vornehmen. \\
			Mehr: Skript-Kinematik (S.11 ff) \& UB5 Bahnplanung (Aufg. 2))
			
\subsection{Denavit-Hartenberg}
\begin{minipage}{19cm}
	\textbf{Ablauf:}
	\begin{enumerate}{\setlength{\itemsep}{0cm}\setlength{\parsep}{0cm} \setlength{\topsep}{0cm}}
      \item Gelenke nummerieren in aufsteigender Reihenfolge. Starten in der Basis mit Nummer null.
      \item Jeden Achskörper mit Koordinatensystem belegen.
      \item Die $z_i$-Koordinatenachse muss mit der i+1 Gelenkachse zusammenfallen.
      \item Die $x_i$-Achse liegt entlang der Normalen zwischen der $z_{i-1}$ und $z_i$-Achse und zeigt vom Gelenk i zum Gelenk i+1.
      \item $y_i$-Achsen vervollständigen mit der Rechten-Hand-Regel. (x:Daumen, y:Zeigfinger, z:Mittelfinger)
      \item Festlegen der DH-Parameter (siehe DH-Parameter) und eintragen in DH-Tabelle.
      \item DH-Matrizen berechnen und miteinander mulitplizieren.
    \end{enumerate}
    \vspace{0.2cm}
\end{minipage}\\

\begin{minipage}{19cm}
	\textbf{Anmerkung Koordinatensysteme:}
	\begin{itemize}\itemsep0pt
      \item $z_i$-Achse muss grundsätzlich mit Bewegungsachse des zugehörigen Achskörper zusammenfallen.
      		Bei Rotationsgelenken gilt die Rechte-Handregel für Drehungen. 
      \item Ursprung des Koordinatensystems im Schnittpunkt der Bewegungsachsen.
    \end{itemize}
    \vspace{0.2cm}
\end{minipage}\\

\begin{minipage}{19cm}
	\textbf{DH-Parameter:}\\ \\
	\begin{tabular}{l l}
  		Linklänge $a_i$ (Fixwert): 				& Für $z_{i-1}$- und $z_i$-Achse wird die gem. Normale mit Länge $a_i$ in $x_i$-Richtung gemessen.\\
  		Linkdrehung $\alpha_{i}$ (Fixwert):		& Drehwinkel um $x_i$-Achse bis $z_{i-1}$- und $z_i$-Achse in gleiche Richtung zeigen.\\
  		Link Offset $d_i$ (Variable):			& Abstand von $x_{i-1}$- und $x_i$-Achse entlang der $z_{i-1}$-Achse.\\
 	 	Gelenkwinkel $\theta_{i}$ (Variable):	& Drehwinkel um $z_{i-1}$-Achse bis $x_{i-1}$- und $x_i$-Achse in gleiche Richtung zeigen.\\
    \end{tabular}
	\vspace{0.5cm}
\end{minipage}\\
\begin{minipage}{19cm}
\textbf{DH-Tabelle:}\\ \\
	\begin{minipage}{10cm}
    	\renewcommand{\arraystretch}{1.1}
			\begin{tabular}{| c | c | c | c | c |}
				\hline
					\textbf{Gelenk Nr.}
					& \textbf{Linklänge $a_i$}
					& \textbf{Linkdrehung $\alpha_{i}$}
					& \textbf{Link Offset $d_i$} 
					& \textbf{Gelenkwinkel $\theta_{i}$}\\
				\hline
					i
					&&&& \\
				\hline
					i+1
					&&&& \\
				\hline
					\ldots
					&&&&\\
				\hline
			\end{tabular}
		\renewcommand{\arraystretch}{1}
		\vspace{0.5cm}
    \end{minipage}
\end{minipage}\\
\begin{minipage}{19cm}
	\textbf{DH-Matrizen:}\\ \\
	$ ^{i-1}_{i}T =
	\begin{bmatrix}
    	cos(\theta_i) & -sin(\theta_i) cos(\alpha_i) &  sin(\theta_i) sin(\alpha_i) & a_i cos(\theta_i)\\
    	sin(\theta_i) &  cos(\theta_i) cos(\alpha_i) & -cos(\theta_i) sin(\alpha_i) & a_i sin(\theta_i)\\
    	0			  &  sin(\alpha_i)				 &  cos(\alpha_i)				& d_i\\
    	0			  &  0							 &  0							& 1\\
    \end{bmatrix}
\qquad
	 ^{0}_{n}T = \prod\limits_{i=1}^{n} \quad ^{i-1}_{i}T(\theta_{i}) = ^{0}_{1}T \cdot ^{1}_{2}T \cdot \ldots \cdot ^{n-1}_{n}T $
\end{minipage}\\\\

\begin{minipage}{3cm}
\textbf{Beispiel:}
\includegraphics[width=4cm]{./bilder/denavit_grafik.png} \\
\end{minipage}
\begin{minipage}{6cm}
$\alpha_{i} \Longrightarrow $ Linkdrehung  \\
$ a_{i} \Longrightarrow $ Linklänge [Achsenabstand] \\
$ d_{i} \Longrightarrow $ Offset \\
$ \Theta_{i} \Longrightarrow $ Gelenkwinkel \\ 
\end{minipage}
\begin{minipage}{8cm}
\includegraphics[width=9cm]{./bilder/denavit_tabelle.png} \\
\includegraphics[width=9cm]{./bilder/denavit_matrix.png} \\
\end{minipage} \\
