\thispagestyle{empty}
\setcounter{page}{0} %Set PageNumber to 0
{\huge README }
\section*{Beschreibung}
Zusammenfassung für Robotik auf Grundlage der Vorlesung HS 17 von Prof. Dr. Agathe Koller-Hodac. \newline
Bei Korrekturen oder Ergänzungen wendet euch an einen der Mitwirkenden.

\subsection*{Buch}
Industrieroboter - Methoden der Steuerung und Regelung \newline
3., neu bearbeitete Auflage\newline
Wolfgang Weber\newline
ISBN: 978-3-446-43355-7

\subsection*{Taschenrechner}
Diverse Prog. für den Ti-Nspire CX CAS findet man unter \href{https://github.com/LMazzole/tinspire}{Github LMazzole/tinspire}
\section*{Modulschlussprüfung}
Erlaubte Hilfsmittel:
\begin{itemize}
    \item Vorlesungsunterlagen (aber keine alten Prüfungen)
    \item Empfohlene Bücher Robotik
    \item Taschenbuch der Physik
    \item Taschenrechner (nicht kommunikationsfähig)
\end{itemize}


\subsection*{Plan und Lerninhalte}
%{\scriptsize 
    \begin{itemize}
        \item Einführung in die Funktion, Arbeitsweise und Programmierung von Robotern und Handlingssystemen
        \item Vermittlung von interdisziplinären Entwicklungsmethoden im Gebiet der Robotik
    \end{itemize}
%}
\vfill
\section*{Contributors}
\begin{tabular}{ll}
    Luca Mazzoleni& luca.mazzoleni@hsr.ch \\ 
\end{tabular} 

{\scriptsize 
    \section*{License}
    \textbf{Creative Commons BY-NC-SA 3.0}
    
    Sie dürfen:
    \begin{itemize}
        \item Das Werk bzw. den Inhalt vervielfältigen, verbreiten und öffentlich
        zugänglich machen.
        \item Abwandlungen und Bearbeitungen des Werkes bzw. Inhaltes anfertigen.
    \end{itemize}
    Zu den folgenden Bedingungen:
    \begin{itemize}
        \item Namensnennung: Sie müssen den Namen des Autors/Rechteinhabers in der von ihm
        festgelegten Weise nennen.
        \item Keine kommerzielle Nutzung: Dieses Werk bzw. dieser Inhalt darf nicht für
        kommerzielle Zwecke verwendet werden.
        \item  Weitergabe unter gleichen Bedingungen: Wenn Sie das lizenzierte Werk bzw. den
        lizenzierten Inhalt bearbeiten oder in anderer Weise erkennbar als Grundlage
        für eigenes Schaffen verwenden, dürfen Sie die daraufhin neu entstandenen
        Werke bzw. Inhalte nur unter Verwendung von Lizenzbedingungen weitergeben,
        die mit denen dieses Lizenzvertrages identisch oder vergleichbar sind.
    \end{itemize}
    Weitere Details: http://creativecommons.org/licenses/by-nc-sa/3.0/ch/
}
%If we meet some day, 
%and you think this stuff is worth it, you can buy me a beer in return.
\clearpage
\pagenumbering{arabic}% Arabic page numbers (and reset to 1)